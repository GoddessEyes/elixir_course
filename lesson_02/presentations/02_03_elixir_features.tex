\documentclass[10pt]{beamer}

\usepackage{fontspec}
\setmainfont{Ubuntu}[]
\setsansfont{Ubuntu}[]
\setmonofont{Ubuntu Mono}[]

\usepackage{graphicx}
\graphicspath{ {../img/} }

\usepackage{listings}
\lstset{
  language=ML,
  keywordstyle=\color{blue},
  backgroundcolor=\color{lightgray}
}

\beamertemplatenavigationsymbolsempty

\title{Свойства Эликсир}

\begin{document}

\begin{frame}
  \frametitle{Свойства Эликсир}
  \begin{itemize}
  \item Мощная система макросов
  \item Высокоуровневый язык
  \item Экосистема и сообщество
  \end{itemize}
\end{frame}

\begin{frame}
  \frametitle{Система макросов}
  Работает на уровне абстрактного синтаксического дерева.
  \par \bigskip
  Позволяет расширять язык и строить Domain~Specific~Languages.
  \par \bigskip
  Примеры: \textbf{Ecto} и \textbf{Phoenix}.
\end{frame}

\begin{frame}
  \frametitle{Высокоуровневый язык}
  \begin{itemize}
  \item Оператор \textbf{pipe}
  \item Макрос \textbf{with}
  \item Абстракции Эликсир
  \item Абстракции Эрланг
  \end{itemize}  
\end{frame}


\begin{frame}[fragile]
  \frametitle{Оператор pipe}
  Несколько вызовов функций:
  \begin{lstlisting}
    a = 42
    b = func1(a)
    c = func2(b)
    d = func3(c)
  \end{lstlisting}
  Можно собрать в цепочку:
  \begin{lstlisting}
    d = func3(func2(func1(42))
  \end{lstlisting}
  С оператором pipe сохраняется естественный порядок:
  \begin{lstlisting}
    d = 42 |> func1() |> func2() |> func3()
  \end{lstlisting}
\end{frame}

\begin{frame}[fragile]
  \frametitle{Макрос with}
  Композиция функций, которые могут вернуть ошибку:
  \begin{lstlisting}
    case func1() do
      {:ok, a} -> 
        case func2() do
          {:ok, b} -> 
            case func3() do
              {:ok, c} -> a + b + c
              error -> error
            end
          error -> error
       end
       error -> error
    end
  \end{lstlisting}
\end{frame}

\begin{frame}[fragile]
  \frametitle{Макрос with}
  С макросом \textbf{with}:
  \begin{lstlisting}
    with(
      a = func1(),
      b = func2(),
      c = func3()
    ) do
      a + b + c
    end
  \end{lstlisting}
  Обработка ошибок в стиле \textbf{Railway programming} (или~\textbf{do-notation}).
\end{frame}

\begin{frame}
  \frametitle{Абстракции Эликсир}
  \begin{itemize}
  \item Коллекции \textbf{Enum}
  \item Ленивые коллекции \textbf{Stream}
  \item Строковые литералы \textbf{Sigil}
  \item Протоколы
  \item \textbf{Task} и \textbf{Agent}
  \end{itemize}
\end{frame}

\begin{frame}
  \frametitle{Абстракции Эрланг}
  \begin{itemize}
  \item Функции высшего порядка
  \item Конструкторы списков
  \item Сопоставление с образцом
  \item \textbf{gen\_server} и \textbf{supervisor}
  \end{itemize}
\end{frame}

\begin{frame}
  \frametitle{Экосистема и сообщество}
  \begin{itemize}
  \item Большое и активное сообщество
  \item Много библиотек и инструментов
  \item Язык и экосистема быстро развиваются
  \item Заимствование идей и инструменов из Эликсир в Эрланг
  \end{itemize}
\end{frame}

\end{document}
