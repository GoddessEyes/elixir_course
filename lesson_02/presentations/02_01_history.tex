\documentclass[10pt]{beamer}
\usepackage{fontspec}
\usepackage{listings}

\setmainfont{Ubuntu}[]
\setsansfont{Ubuntu}[]
\setmonofont{Ubuntu Mono}[]

\usetheme{Singapore}
\usecolortheme{dove}
\beamertemplatenavigationsymbolsempty
\setbeamertemplate{headline}{}

\lstset{
  language=ML,
  keywordstyle=\color{blue},
  backgroundcolor=\color{lightgray}
}

\title{Немного истории}

\begin{document}
\maketitle

\begin{frame}
\frametitle{Немного истории}
\centering
Предыстория: начало ХХ века.
\par \bigskip
История Эрланг: c 1985 по настоящее время.
\par \bigskip
История Эликсир: c 2012 по настоящее время.
\end{frame}


\begin{frame}
\frametitle{Агнер Краруп Эрланг}
\centering
Датский математик, статистик и инженер.
\par \bigskip
Научный подход к изучению трафика в телекоммуникационных системах.
\par \bigskip
Автор "Теории массового обслуживания".
\end{frame}

\begin{frame}
\frametitle{Теория массового обслуживания}
\centering
Она же теория очередей (Queueing theory).
\par \bigskip
Математическая модель для оценки пропускной способности телекоммуникационных сетей.
\end{frame}

применяется в телекоммуникационных системах,
и гораздо шире:
в управлении автомобильным и воздушным движением,
на конвейерном производстве,
в логистике,
при проектировании складов, магазинов и больниц.

Язык [Эрланг](https://www.erlang.org/) родился в шведской компании Эрикссон (Ericsson)
крупного поставщика телекомуникационного оборудования и услуг.

Для этой индустрии характерны
сложное оборудование,
сложный софт,
большой траффик
и жесткие требования по доступности сервиса.

Ericsson’s Computer Science Laboratory. 
задача найти более эффективные средства
разработки софта для железа и сервисов компании.

прототипы телеком-приложений на разных языках:
- функциональные языки ML и Miranda;
- многопоточные языки ADA, Modula и Chill;
- логический язык Prolog;
- объектно-ориентированный Smalltalk.

Разработчики пришли к выводу, что ни один язык не имеет нужных возможностей.
И главная проблема была в реализации многопоточности на нужном уровне.
В итоге лаборатория решила разработать свой язык программирования.

В 2002 году был начат проект [ejabberd](https://www.ejabberd.im)
первый крупный open source проект на Эрланг.
Он стал основой для большинства IM (Instant Messaging) систем,
в т.ч. для широко известного [WhatsApp](https://en.wikipedia.org/wiki/WhatsApp).

В 2006 году в появилась поддержка [симметричной мультипроцессорности (SMP)](https://en.wikipedia.org/wiki/Symmetric_multiprocessing).
Эрланг научился эффективно использовать все имеющиеся в системе процессорные ядра.

А у IT-индустрии появилась потребность разрабатывать многопоточные программы,
эффективно использующие несколько процессоров.
Делать это на популярных языках программирования было трудно,
и возник интерес к функциональному программированию вообще, и к Эрланг в частности.

Этот интерес проявился в двух направлениях:
- стали шире использоваться ФП языки;
- популярные языки начали заимствовать идеи ФП и реализовывать их у себя.

[Эликсир](https://elixir-lang.org/) создан в 2012 году в компании Plataformatec.
Его автор -- Жозе Валим,
был одним из основных разработчиков фреймворка Ruby on Rails
и сооснователем компании Plataformatec.

Эликсир позаимствовал идеи из Ruby, Clojure и Эрланг.
Большое влияние оказали Ruby и фрейморк Ruby on Rails.
Система макросов заимствована из Clojure.
Ну и, конечно, Эликсир унаследовал все возможности Эрланг и его виртуальной машины.

Первая версия языка вышла в 2014 году.
и первая версия фреймворка Phoenix.

\end{document}
