\chapter{Решаем задачу FizzBuzz}

Начнем знакомство с Эликсир на примере решения задачи \href{https://ru.wikipedia.org/wiki/Fizz_buzz}{FizzBuzz}

\textit{Напишите программу, которая выводит на экран числа от 1 до 100. При этом вместо чисел, кратных трем, программа должна выводить слово «Fizz», а вместо чисел, кратных пяти — слово «Buzz». Если число кратно и 3, и 5, то программа должна выводить слово «FizzBuzz».}

Это простая задача позволит познакомиться со многими важными элементами языка:
\begin{itemize}
\item модули и функции;
\item генерация списка с помощью \textbf{Range};
\item итерация по списку с помощью \textbf{Enum.each};
\item условые переходы с помощью \textbf{cond do};
\item охранные выражения (\textbf{guards});
\item вывод на консоль;
\item оператор \textbf{pipe};
\item и модульные тесты (\textbf{unit tests}).
\end{itemize}

\section{Шаг 1. Простая реализация задачи.}

Создаем модуль \textbf{FizzBuzz01} и в нем две функции \textbf{main} и \textbf{fizzbuzz}. 

\lstinputlisting[]{./lesson_01/lib/fizz_buzz_01.exs}

В функции \textbf{main} мы генерируем последовательность от 1 до 100. Конструкция \textbf{1..100} -- это генератор последовательности, он называется \textbf{Range}. Затем с помощью \textbf{Enum.each} мы применяем функцию \textbf{fizzbuzz} к каждому элементу.

\textbf{fizzbuzz} использует конструкцию \textbf{cond do} и охранные выражения (guards) чтобы проверить условия делимости на 3 и на 5. Первое охранное выражение выполняется, если \textbf{n} делится на 3 и на 5. Второе охранное выражение выполняется, если \textbf{n} делится на 3. Третье, если \textbf{n} делится на 5. И последнее, четвертое охранное выражение выполняется всегда, так как оно представлено просто значением \textbf{true}. 

Функция \textbf{rem}, как не трудно догадаться, возвращает остаток от деления.

\textbf{cond do} проверяет выражения по очереди, и выполняет только одну ветку кода, соответствующую первому истинному выражению. \textbf{IO.puts} выводит нужное значение на стандартный вывод.

Соберем и запустим наш код:
TODO


\section{Шаг 2. Отделяем вывод на консоль от логики.}

Функция \textbf{fizzbuzz} выводит сообщения на консоль, что не очень хорошо с точки зрения функционального программирования. Разделим нашу задачу на две:
