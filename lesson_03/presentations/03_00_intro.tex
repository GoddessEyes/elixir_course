\documentclass[10pt]{beamer}

\usepackage{fontspec}
\setmainfont{Ubuntu}[]
\setsansfont{Ubuntu}[]
\setmonofont{Ubuntu Mono}[]

\usepackage{graphicx}
\graphicspath{ {../img/} }

\usepackage[absolute,overlay]{textpos} % [showboxes]

\beamertemplatenavigationsymbolsempty

\title{Типы данных}

\begin{document}

\begin{frame}
  \frametitle{Типы данных}
  Неофициальная классификация
  \par \bigskip
  \begin{itemize}
    \item Числа (Number)
    \item Булевые значения (Boolean)
    \item Атомы (Atom)
    \item Коллекции (Collections)
    \item Системные типы (System Types)
    \item Составные типы (Complex Types)
    \item Пользовательские типы (Data Model)
  \end{itemize}
\end{frame}

\begin{frame}
  \frametitle{Типы данных}
  Числа (Number):
  \begin{itemize}
    \item Целые числа (Integer)
    \item Числа с плавающей точкой (Float)
  \end{itemize}
  \par \bigskip
  Булевые значения (Boolean):
  \begin{itemize}
    \item true, false, nil
  \end{itemize}
  \par \bigskip
  Атомы (Atom):
  \begin{itemize}
    \item :ok, :error, MyModule
  \end{itemize}
\end{frame}

\begin{frame}
  \frametitle{Коллекции (Collections)}
  \begin{itemize}
    \item Кортежи (Tuple)
    \item Списки (List)
    \item Словари (Map)
    \item Строки (String), они же бинарные данные (Binary)
    \item Диапазоны (Range)
  \end{itemize}
\end{frame}

\begin{frame}
  \frametitle{Системные типы (System Types)}
  \begin{itemize}
    \item Идентификаторы процессов (Pid)
    \item Идентификаторы портов (Port)
    \item Уникальные идентификаторы (Reference)
    \item Функции (Function)
  \end{itemize}
\end{frame}

\begin{frame}
  \frametitle{Составные типы (Complex Types)}
  \begin{itemize}
    \item IO-Списки (IO List)
    \item Списки пар ключ-значение (Keyword List)
  \end{itemize}
\end{frame}

\begin{frame}
  \frametitle{Пользовательские типы (Data Model)}
  \begin{itemize}
    \item Структуры (Struct)
    \item Специальные строки (Sigil): Date, Time, DateTime, RE
  \end{itemize}
\end{frame}

\end{document}
