\chapter{Типы данных}

Любой язык программирования начинается с базовых типов данных и операций над ними.

Из базовых типов строятся более сложные типы, которые тоже имеют свой набор операций.

Наконец, из базовых и сложных типов можно создавать новые, пользовательские типы данных, и реализовывать новые операции над ними.

В этом уроке мы начнем знакомиться с базовыми типами.

number
  integer
  float

boolean

atom

collections
  tuple
  list
  map
  string/binary

system types
  pid
  port
  reference
  function

complex types
  io list
  keyword list
  range
  sigil: Date, Time, DateTime, RE
