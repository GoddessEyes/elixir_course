\chapter{Интерактивная консоль}

Посмотрим на интерактивную консоль \name{iex} -- второй по важности, после редактора, инструмент разработчика.

Многие языки программирования имеют \name{REPL} консоль. Она дает возможность работать интеративно: запускать небольшие куски кода и сразу же видеть результ. REPL означает Read, Eval, Print, Loop. То есть, консоль читает код, выполняет его, выводит результат, и опять читает код.

Мы уже начали активно пользоваться консолью когда реализовывали \codename{FizzBuzz}. Теперь познакомимся с ней поближе:

\begin{lstlisting}[caption=Запуск консоли, language=ElixirShell, style=elixir-shell]
$ iex
Erlang/OTP 24 [erts-12.2.1] [source] [64-bit] [smp:12:12] [ds:12:12:10] [async-threads:1] [jit]

Interactive Elixir (1.13.3) - press Ctrl+C to exit (type h() ENTER for help)
iex>
\end{lstlisting}

Мы видим версию Эрланг: \code{Erlang/OTP 24}, версию Erlang Runtime System: \code{erts-12.2.1} и версию Эликсир: \code{Elixir (1.13.3)}.

\section{Компиляция}

iex-консоль имеет много встроенных функций. Самые популярные из них имеют однобуквенные называния.

Например, функция \name{c} компилирует и загружает код:

\begin{lstlisting}[language=ElixirShell, style=elixir-shell]
iex> c "lib/fizz_buzz_02.exs"
[FizzBuzz02]
B>
\end{lstlisting}

Другая популярная функция -- это \name{h}, которая показывает справку по функциям и модулям Эликсир. Посмотрим справку по функции \name{c}:

\begin{lstlisting}[language=ElixirShell, style=elixir-shell]
iex> h c
def c(files, path \\ :in_memory)

Compiles the given files.
...
\end{lstlisting}

Мы видим, что функция \name{c} компилирует один или несколько файлов, загружает их в память консоли, или сохраняет сгенерированный байт-код по указаному пути.

Если мы вносим изменения в код, то с помощью функции \name{r} мы можем перекомпилировать модуль заново и загрузить свежую версию.

\begin{lstlisting}[language=ElixirShell, style=elixir-shell]
iex> r FizzBuzz02
warning: redefining module FizzBuzz02 (current version defined in memory)
  lib/fizz_buzz_02.exs:1

{:reloaded, [FizzBuzz02]}
\end{lstlisting}

Таким образом удобно вести интерактивную разработку -- писать код и тут же проверять в консоли, как он работает.

Для больших проектов неудобно перекомпилировать по-отдельности каждый измененный модуль. Поэтому в консоли есть функция \name{recompile}, которая пересобирает весь проект. Однако, для этого нужно иметь полноценный проект с правильной структурой кода, а не просто несколько модулей.

\section{Автодополнение}

Для кода, который скомпилирован и загружен в консоль, работает автодополнение модулей и функций:

\begin{lstlisting}[language=ElixirShell, style=elixir-shell]
iex> Fizz.<tab>
iex> FizzBuzz02.<tab>
iex> FizzBuzz02.m<tab>
iex> FizzBuzz02.main
\end{lstlisting}

\section{Интроспекция}

Еще одна полезная функция, \name{i} -- это интроспеция, которая показывает информацию о значениях и переменных:

\begin{lstlisting}[language=ElixirShell, style=elixir-shell]
iex> i 42
Term
  42
Data type
  Integer
...

iex> i 3.14
Term
  3.14
Data type
  Float
...

iex> my_str = "Hello"
iex> i my_str
Term
  "Hello"
Data type
  BitString
Byte size
  5
Description
  This is a string: a UTF-8 encoded binary ...

iex> my_list = [1, 2, 3]
iex> i my_list
Term
  [1, 2, 3]
Data type
  List
...
\end{lstlisting}

\section{История}

Как и большинство других консолей, iex-консоль сохраняет историю команд и позволяет перемещатсья по ней с помощью горячих клавиш \code{Ctrl-p} и \code{Ctrl-n} или стрелками.

Также консоль позволяет обратиться к результатам предыдущих команд с помощью функции \name{v}. Она принимает номер строки и возвращает результат команды из этой строки. Её также можно вызвать без аргументов, тогда она вернет результат предыдущей команды:

\begin{lstlisting}[language=ElixirShell, style=elixir-shell]
iex> 42 * 2
84
iex> v
84
iex> v 9
84
iex> 20 + 30
50
iex> v
50
iex> v 12
50
iex> v 9
84
\end{lstlisting}

\section{Справка}

Рассмотрим подробнее функцию \name{h}. Если вызывать ее без аргументов, то мы увидим справку по самой консоли iex.

Аргументом можно передать имя модуля, функции или макроса. Тогда консоль покажет справку по ним:
\begin{lstlisting}[language=ElixirShell, style=elixir-shell]
iex> h
iex> h Map
iex> h Map.get
iex> h Map.<tab>
iex> h Map.fetch
\end{lstlisting}

В каждый модуль при компиляции автоматически добавляется функция \code{module\_info}, которая возращает некоторые метаданные об этом модуле. Это справедливо как для библиотечных модулей, так и для модулей, которые мы создаем сами:

\begin{lstlisting}[language=ElixirShell, style=elixir-shell]
iex> Map.module_info
[
  module: Map,
  exports: [
    __info__: 1,
    delete: 2,
    equal?: 2,
    fetch: 2,
    fetch!: 2,
...

iex> FizzBuzz02.module_info
[
  module: FizzBuzz02,
  exports: [__info__: 1, fizzbuzz: 1, main: 0, module_info: 0, module_info: 1],
  attributes: [vsn: [19328415140430776990959847375304151410]],
  compile: [
...
\end{lstlisting}


\section{Справка по собственным модулям}

Мы можем добавить документацию в собственные модули, и эта документация будет доступна в iex-консоли.

Документация для модуля задаётся с помощью аттрибута \code{@moduledoc}, а для функции с помощью аттрибута \code{@doc}:

\begin{lstlisting}[language=Elixir, style=elixir]
defmodule FizzBuzz04 do

  @moduledoc """
  FizzBuzz is a simple task to show basic usage of Elixir.
  """
  ...

  @doc "Produces list of strings for numbers from 1 to 100."
  @spec fizzbuzz_100() :: [String.t]
  def fizzbuzz_100() do
    1..100
    |> Enum.map(&fizzbuzz/1)
  end
  ...
\end{lstlisting}
Однако это не работает, если модуль просто скомпилирован из консоли:

\begin{lstlisting}[language=ElixirShell, style=elixir-shell]
iex> c "lib/fizz_buzz_04.exs"
iex> h FizzBuzz04
FizzBuzz04 was not compiled with docs
\end{lstlisting}

Нужно, чтобы скомпилированный байткод был сохранён на диске. Для этого функцию \name{c} вызываем с двумя аргументами, кроме файла исходников ещё указываем путь, куда сохранять beam-файл:

\begin{lstlisting}[language=ElixirShell, style=elixir-shell]
iex> c("lib/fizz_buzz_04.exs", ".")
iex> h FizzBuzz04
FizzBuzz04

FizzBuzz is a simple task to show basic usage of Elixir.

iex> h FizzBuzz04.fizzbuzz_100

def fizzbuzz_100()

@spec fizzbuzz_100() :: [String.t()]

Produces list of strings for numbers from 1 to 100.

iex> h FizzBuzz04.fizzbuzz

 def fizzbuzz(n)

@spec fizzbuzz(integer()) :: String.t()

Produces string result for a single number.
\end{lstlisting}

\section{runtime\_info}

Рассмотрим еще функцию \code{runtime\_info}. Длинное название этой функции подразумевает, что она не используется так же часто, как \name{c}, \name{h}, \name{i} или \name{v}.

Функция выводит полезную информацию о текущем состоянии виртуальной машины: версии Эликсир и Эрлаг, количество планировщиков, потребление памяти, количество процессов и др.

\begin{lstlisting}[language=ElixirShell, style=elixir-shell]
iex> runtime_info
## System and architecture

Elixir version:     1.13.4
Erlang/OTP version: 23
...

## Memory

Total:              20 MB
Atoms:              328 KB
...

## Statistics / limits

Uptime:             1 minutes and 40 seconds
...

iex> runtime_info :system

## System and architecture

Elixir version:     1.13.4
Erlang/OTP version: 23
ERTS version:       11.1
Compiled for:       x86_64-unknown-linux-gnu
Schedulers:         8
Schedulers online:  8
...

iex> runtime_info :memory

## Memory

Total:              20 MB
Atoms:              328 KB
Binaries:           111 KB
Code:               7 MB
ETS:                510 KB
Processes:          4 MB
...
\end{lstlisting}

Как видно, iex-консоль очень полезна. Опытные разработчики активно ей пользуются.
